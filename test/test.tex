\documentclass{article}
\usepackage[dutch]{babel}
\usepackage[utf8]{inputenc}
\usepackage{longtable}
\usepackage[a4paper,margin=2.5cm]{geometry}
\usepackage[backend=biber,style=nature]{biblatex}
\DeclareLanguageMapping{dutch}{dutch-apa}
\addbibresource{references.bib}

% Hyperref moet als laatste package komen
\usepackage[hidelinks]{hyperref}  % hidelinks verwijdert de rode boxes rond links
% Als je de rode boxes wilt behouden, gebruik dan gewoon \usepackage{hyperref}

% Optioneel: pas de kleur van de links aan
\hypersetup{
    colorlinks=true,     % false: boxed links; true: colored links
    linkcolor=black,     % color of internal links
    citecolor=blue,      % color of citations
    urlcolor=blue        % color of external urls
}
\title{Logboek PWS}
\author{Groep}
\date{2024}
  
\begin{document}
\maketitle


\section*{Groepsactiviteiten}
\begin{longtable}{|p{0.15\textwidth}|p{0.1\textwidth}|p{0.1\textwidth}|p{0.55\textwidth}|}
    \hline
    \textbf{Datum} & \textbf{Tijd} & \textbf{Plaats} & \textbf{Activiteiten + Resultaten}                                                                                                                                                                                                                                                                                   \\
    \hline
    02-07-2024     & 3 uur         & School          & Onderzoek over AI, we leerden alle drie over reinforced learning, wat die inhield en wat we interessant vonden. PWS presentatie, deze heeft goed geholpen om te begrijpen wat we moeten doen. De bronnenlijst is gemaakt, deze duurt het langst. Hier zijn de bronnen ingezet waarvan wij denken dat ze handig zijn. \\
    \hline
    28/08/2024     & 1 uur         & School          & Overleg tijdens les                                                                                                                                                                                                                                                                                                  \\
    \hline
    04/09/2024     & 1 uur         & School          & Overleg tijdens les en taakverdeling gedeeltelijk geregeld                                                                                                                                                                                                                                                           \\
    \hline
    11/09/2024     & 1 uur         & School          & Inlezen onderwerp                                                                                                                                                                                                                                                                                                    \\
    \hline
    25/09/2024     & 1 uur         & School          & Overleg indeling schrijven                                                                                                                                                                                                                                                                                           \\
    \hline
    04/09/2024     & 1 uur 15 min  & School          & Inlezen onderwerp. Onderling plan van aanpak besproken                                                                                                                                                                                                                                                               \\
    \hline
    11/09/2024     & 1 uur         & School          & Taken en deadlines besproken. Taken verdeeld voor volgende pws uur. Verder ingelezen over onderwerp                                                                                                                                                                                                                  \\
    \hline
    02/10/2024     & 1 uur         & School          & Bronnenlijst overleg                                                                                                                                                                                                                                                                                                 \\
    \hline
    09/10/2024     & 1 uur         & School          & Voorwoord maken                                                                                                                                                                                                                                                                                                      \\
    \hline
    16/10/2024     & 1 uur         & School          & Onderzoek AI                                                                                                                                                                                                                                                                                                         \\
    \hline
    23/10/2024     & 1 uur         & School          & Onderzoek RL                                                                                                                                                                                                                                                                                                         \\
    \hline
    06/11/2024     & 1 uur         & School          & Inlezen                                                                                                                                                                                                                                                                                                              \\
    \hline
    13/11/2024     & 1 uur         & School          & Verder inlezen                                                                                                                                                                                                                                                                                                       \\
    \hline
    20/11/2024     & 1 uur         & School          & Layout maken                                                                                                                                                                                                                                                                                                         \\
    \hline
    27/11/2024     & 1 uur         & School          & Meer samenvatten                                                                                                                                                                                                                                                                                                     \\
    \hline
    04/12/2024     & 1 uur         & School          & Overleg                                                                                                                                                                                                                                                                                                              \\
    \hline
\end{longtable}
\section*{Matthijs}
\begin{longtable}{|p{0.15\textwidth}|p{0.1\textwidth}|p{0.1\textwidth}|p{0.55\textwidth}|}
    \hline
    \textbf{Datum} & \textbf{Tijd} & \textbf{Plaats}         & \textbf{Activiteiten + Resultaten}                                                                                                                                                                                                                                    \\
    \hline
    06-07-2024     & 1 uur         & Thuis                   & Eerste Stanfords CS234 college bekeken uit de winter van 2019 \cite{stanford_cs234}. Het kopje Definitie van Reinforcement Learning geschreven. (Later besloten dit in de inleiding te zetten)                                                                      \\
    \hline
    29/07/2024     & 3 uur         & Thuis                   & Snake spel geschreven in python en github aangemaakt en begonnen met Q-learning te implementeren \cite{snake_game} \cite{python_docs}                                                                                                                             \\
    \hline
    30/07/2024     & 2 uur         & Thuis                   & Q-learning geïmplementeerd. Maar de agent leert nog slecht.                                                                                                                                                                                                           \\
    \hline
    31/07/2024     & 3 uur         & Thuis                   & Q-learning hyperparameters uitgetest en optimale gevonden.                                                                                                                                                                                                            \\
    \hline
    03/08/2024     & 2 uur         & Thuis                   & Kennis opgedaan over machine learning en andere types dan reinforcement learning en hoe het verschilt van super- en unsupervised learning \cite{ml_types}                                                                                                           \\
    \hline
    05/08/2024     & 3 uur         & Thuis                   & Begin gemaakt een theoretisch kader \cite{spinning_up}                                                                                                                                                                                                              \\
    \hline
    06/08/2024     & 2 uur         & Thuis                   & Theoretisch kader afgemaakt.                                                                                                                                                                                                                                          \\
    \hline
    08/08/2024     & 2 uur         & Thuis                   & LaTeX geleerd en eerste layout gemaakt van het PWS met alle kopjes.                                                                                                                                                                                                   \\
    \hline
    1/09/2024      & 3 uur         & Thuis                   & Voorstel gemaakt                                                                                                                                                                                                                                                      \\
    \hline
    10/09/2024     & 3 uur         & Thuis, online met groep & Voorwoord gemaakt, overleg over de aanpak en het algemene idee.                                                                                                                                                                                                       \\
    \hline
    20/09/2024     & 2 uur         & Thuis                   & Het Q-learning algoritme tijdsefficiënter gemaakt met numpy en een bug gefixt in snake zodat er nu geen appel kan spawnen in de slang zelf \cite{numpy_docs}                                                                                                        \\
    \hline
    27/09/2024     & 4 uur         & Thuis                   & Script gemaakt voor resultaten van Q-learning in grafiek (5000 woorden, gebruik van Matplotlib). Hyperparameters uitgetest en optimale gevonden. Elke training kostte ongeveer 15 minuten. (gemiddeld score van beste waardes was 60 appels) \cite{matplotlib_docs} \\
    \hline
    24/10/2024     & 4 uur         & Thuis                   & Script gemaakt voor DQN algoritme met PyTorch \cite{pytorch_docs} \cite{pytorch_rl}                                                                                                                                                                               \\
    \hline
    17/11/2024     & 3 uur         & Thuis                   & Hoofdstuk Kenmerken van specifieke algoritmes begonnen. Introductie geschreven en DQN Algoritme vertaald \cite{delayed_rewards} \cite{multi_agent_rl}                                                                                                             \\
    \hline
    24/11/2024     & 3 uur         & Thuis                   & Sectie Deep Q-Network begonnen. Algoritme vertaald, proces beschreven \cite{dqn_nature}                                                                                                                                                                             \\
    \hline
    28/11/2024     & 1 uur         & Thuis                   & Flowchart van Deep Q-Network algoritme gemaakt en verder geschreven                                                                                                                                                                                                   \\
    \hline
    29/11/2024     & 3 uur         & Thuis                   & Kopje Neuraal Netwerk geschreven en diagram van neuraal netwerk in DQN gemaakt \cite{nn_math}                                                                                                                                                                       \\
    \hline
    31/11/2024     & 3 uur         & Thuis                   & Onderzoek over Deep Deterministic Policy Gradient en algoritme vertaald \cite{ddpg} \cite{silver_dpg}                                                                                                                                                             \\
    \hline
    2/12/2024      & 2 uur         & Thuis                   & Proces van DDPG geschreven.                                                                                                                                                                                                                                           \\
    \hline
    3/12/2024      & 2 uur         & Thuis                   & Actor-Critic model geschreven en figuur Flowchart van Actor Critic model gemaakt en Toepassingen geschreven.                                                                                                                                                          \\
    \hline
    4/12/2024      & 2 uur         & Thuis                   & Onderzoeksmethoden geschreven.                                                                                                                                                          \\
    \hline
    5/12/2024      & 1 uur         & Thuis                   & Puntes op de i gezet.                                                                                                                                                          \\
    \hline
\end{longtable}

\section*{Thom}
\begin{longtable}{|p{0.15\textwidth}|p{0.1\textwidth}|p{0.1\textwidth}|p{0.55\textwidth}|}
    \hline
    \textbf{Datum} & \textbf{Tijd} & \textbf{Plaats}         & \textbf{Activiteiten + Resultaten}                             \\
    \hline
    03/09/2024     & 2 uur         & Thuis                   & Inlezen over het onderwerp \cite{wiki_rl}                    \\
    \hline
    04/09/2024     & 2 uur         & Thuis                   & Inlezen over het onderwerp \cite{scribbr_rl}                 \\
    \hline
    05/09/2024     & 2 uur         & Thuis                   & Inlezen over het onderwerp \cite{geeksforgeeks_rl}           \\
    \hline
    06/09/2024     & 2 uur         & Thuis                   & Inlezen over het onderwerp \cite{oracle_rl}                  \\
    \hline
    10/09/2024     & 3 uur         & Thuis, online met groep & Voorwoord maken, overleg over het idee                         \\
    \hline
    26/09/2024     & 2,5 uur       & Thuis                   & Layout maken van het verslag, verder inlezen over onderwerp    \\
    \hline
    01/10/2024     & 3 uur         & Thuis                   & Matthijs' theoretisch kader verbeterd                          \\
    \hline
    02/10/2024     & 3 uur         & Thuis                   & Uitleg theoretisch kader                                       \\
    \hline
    04/10/2024     & 3 uur         & Thuis                   & Herschrijven theoretisch kader                                 \\
    \hline
    06/10/2024     & 2 uur         & Thuis                   & Antwoorden deelvraag 1                                         \\
    \hline
    07/10/2024     & 3 uur         & Bij pepijn              & Inleiding herschreven. Plan van aanpak gemaakt.                \\
    \hline
    08/11/2024     & 3 uur         & Thuis                   & Herschrijven van tekst / nalezen versie voor controle moment 2 \\
    \hline
    03/12/2024     & 3 uur         & Thuis                   & Afmaken plaatjes maken                                         \\
    \hline
    04/12/2024     & 2 uur         & Thuis                   & Afmaken PWS voor controlemoment                                \\
    \hline
    05/12/2024     & 2 uur         & Thuis                   & Puntjes op i zetten                                            \\
    \hline
\end{longtable}
\newpage
\section*{Pepijn}
\begin{longtable}{|p{0.15\textwidth}|p{0.1\textwidth}|p{0.1\textwidth}|p{0.55\textwidth}|}
    \hline
    \textbf{Datum} & \textbf{Tijd} & \textbf{Plaats}         & \textbf{Activiteiten + Resultaten}                                                                                         \\
    \hline
    30/08/2024     & 2 uur         & Thuis                   & Ingelezen onderwerp \cite{ai_basics} \cite{datacamp_ai}                                                                \\
    \hline
    1/09/2024      & 1 uur         & Thuis                   & Onderzoeksplan en -overzicht gemaakt(van het voorstel)                                                                     \\
    \hline
    03/09/2024     & 3 uur         & Thuis                   & Ingelezen over het onderwerp \cite{analytics_vidhya_rl} \cite{rl_guide}                                                \\
    \hline
    04/09/2024     & 1 uur         & Thuis                   & Inlezen over q-learning \cite{q_learning_tutorial}                                                                       \\
    \hline
    07/09/2024     & 3,5 uur       & Thuis                   & Inlezen over het onderwerp en Deep q learning \cite{dqn_guide} \cite{sutton_barto}                                     \\
    \hline
    9/09/2024      & 3 uur         & Thuis                   & Ingelezen over het onderwerp \cite{dqn_paper}                                                                            \\
    \hline
    10/09/2024     & 3 uur         & Thuis, online met groep & Voorwoord gemaakt, overleg over de aanpak en het algemene idee.                                                            \\
    \hline
    07/10/2024     & 3 uur         & Thuis                   & Inleiding herschreven en lay-out van andere delen van het pws verbeterd. Plan van aanpak gemaakt voor de rest van het pws. \\
    \hline
    26/09/2024     & 3 uur         & Thuis                   & Inlezen over speleigenschappen \cite{pmlr-v119-cobbe20a} en begonnnen met het schrijven van deelvraag 2.                 \\
    \hline
    29/10/2024     & 3 uur         & Thuis                   & Verder gewerkt aan deelvraag 2 en bijna volledig uitgewerkt                                                                \\
    \hline
    30/10/2024     & 3 uur         & Thuis                   & Deelvraag 2 afgeschreven                                                                                                   \\
    \hline
    1/11/2024      & 2 uur         & Thuis                   & Alles nagelezen wat tot nu toe was gemaakt en de lay-out erg verbeterd.                                                    \\
    \hline
    08/11/2024     & 3 uur         & bij Thom                & Herschrijven van tekst /nalezen versie voor controle moment 2                                                              \\
    \hline
    26/11/2024     & 0,5 uur       & Thuis                   & Inleiding anders verwoord en fouten uit het pws gehaald                                                                    \\
    \hline
    03/12/2024     & 3 uur         & Thuis                   & Begonnen met deelvraag 3                                                                                                   \\
    \hline
    04/12/2024     & 2 uur         & Thuis                   & Deelvraag 3 afgemaakt en verbeterd                                                                                         \\
    \hline
    05/12/2024     & 3 uur         & Thuis                   & PWS nagelezen spelling verbeterd bronnotatie geregeld en puntjes op de i gezet.                                            \\
    \hline
\end{longtable}

\printbibliography[title=Literatuurlijst]

\end{document}